\documentclass[letterpaper]{report}
\usepackage{bytefield}

\begin{document}
\title{Glacial Microarchitecture}
\author{Eric Smith}
\date{October 28, 2019}
\maketitle

\chapter{Introduction}

This document describes the microarchitecture of Glacial, a highly
vertically-microcoded processor. When running the standard microcode,
Glacial implements a RISC-V RV32I processor. However, the Glacial
microarchitecture may be useful for other applications unrelated to
RISC-V.

In the rest of this document, the use of the name "Glacial" will be
in reference to the Glacial microarchitecture, without regard to
the normal microcode implementing the RISC-V architecture.

\section{Limitations}

Glacial was designed to use minimal hardware resources, e.g., FPGA
logic elements, and was designed to support hand-written assembly code,
rather than a high-level language. It is missing many features
commonly present in other processor architectures. For example:

\begin{itemize}
  \item There is an adc instruction (add with carry), but no subtract
instruction, because subtraction can be accomplished by complementing
the subtrahend then adding.

  \item There are instructions for logical and and exclusive or, but not for
logical or. By De Morgan's law, the logical and and exclusive or
instructions can be used to accomplish a logical or function.

  \item There is a postincrement address mode, but no corresponding predecrement
mode

  \item Glacial has a single memory address space of 64 KiB. Glacial can only
execute instructions from the first 4 KiB of the memory; the remaining
60 KiB is used for data (or macroinstructions) only.

  \item The X index register, and instructions using absolute addressing of data,
can only address the first 256 bytes of memory. The rest of memory can
only be accessed as data by use of the Y index register.

  \item There is only one level of subroutine stack.
\end{itemize}

\section{Simple architectural enhancements}

Several relatively simple architectural enhancements could be made,
which might make Glacial more suitable for other applications:

\begin{itemize}

  \item More index registers could be added. The instruction encoding for the
indexed addressing modes currently does not use bits 7..1 of the
instruction.

  \item The address space for data could be increased, e.g., to 16 MiB
or 4 GiB, by making the Y index register (and/or other added index
registers) wider.

  \item The size of the return stack could be increased to allow
multiple levels of subroutine calls.

  \item If it is desired to use ROM starting at address 0, the X index
register could have a fixed but non-zero high byte to address
RAM. This would be similar to the 6502 microprocessor's stack pointer,
which is 8 bits, but has a high byte of 0x01.

\end{itemize}


\chapter{Programmer's Model}

Glacial has six programmer-visible registers:

\bigskip

\begin{tabular}{ | c | c | }
  \hline
  register        & bits \\
  \hline \hline
  program counter & 12 (LSB always zero) \\
  \hline
  return address  & 12 (LSB always zero) \\
  \hline
  x index         & 8 \\
  \hline
  y index         & 16 \\
  \hline
  accumulator     & 8 \\
  \hline
  carry flag      & 1 \\
  \hline
\end{tabular}

\bigskip

Glacial has ten basic instructions, though one of them, opr (operate)
performs various miscellaneous functions:

\bigskip

\begin{tabular}{ | c | c | }
  \hline
  mnemonic & description \\
  \hline \hline
  opr      & operate (miscellaneous functions) \\
  \hline
  store    & store the accumulator into a memory byte \\
  \hline
  load     & load a byte from memory into the accumulator \\
  \hline
  and      & logically and a byte from memory into the accumulator \\
  \hline
  xor      & logically exclusive or a byte from memory into the accumulator \\
  \hline
  adc      & add a byte from memory and carry into the accumulator \\
  \hline
  jump     & unconditional jump \\
  \hline
  call     & unconditional subroutine call \\
  \hline
  skb      & skip next instruction, conditional on the state of an accumulator bit \\
  \hline
  br       & conditional branch \\
  \hline
\end{tabular}

\bigskip

Glacial has five basic memory addressing modes, not all of which are
applicable to all instructions:

\bigskip

\begin{itemize}
  \item immediate 8-bit
  \item absolute 8-bit
  \item index register
  \item index register with postincrement
  \item branch, absolute 13-bit address, even addresses only
\end{itemize}

\chapter{Instruction Formats and Memory Addressing Modes}

\section{immediate 8-bit}

The immediate addressing mode embeds an 8-bit operand direction in the
instruction.

\bigskip

\begin{bytefield}[endianness=big,bitwidth=1.5em]{16}
  \bitheader{0-15} \\
  \bitbox{8}{opcode} &
  \bitbox{8}{immediate operand}
\end{bytefield}

\section{absolute 8-bit}

The absolute addressing mode allows access to any byte within the first 256
bytes of memory (0x0000 through 0x00ff) by a fixed address embedded in the
instruction.

\bigskip

\begin{bytefield}[endianness=big,bitwidth=1.5em]{16}
  \bitheader{0-15} \\
  \bitbox{8}{opcode} &
  \bitbox{8}{absolute address}
\end{bytefield}

\section{index register}

The indexed addressing mode allows access to the byte of memory pointed to
by an index register, x or y. The x index register can only point
to the first 256 bytes of memory (0x0000 through 0x00ff).

\bigskip

\begin{bytefield}[endianness=big,bitwidth=1.5em]{16}
  \bitheader{0-15} \\
  \bitbox{8}{opcode} &
  \bitboxes*{1}{0000000} &
  \bitbox{1}{idx}
\end{bytefield}

\bigskip

The idx field value is 0 for the x index register, and 1 for the y index
register.

\section{index register autoincrement}

The index register autoincrement addressing mode allows access to the byte
of memory pointed to by an index register, x or y, and after the access,
increments the index register. The x index register can only point
to the first 256 bytes of memory (0x0000 through 0x00ff), and if incremented
past 0xff, x will wrap to 0x0000.

\bigskip

\begin{bytefield}[endianness=big,bitwidth=1.5em]{16}
  \bitheader{0-15} \\
  \bitbox{8}{opcode} &
  \bitboxes*{1}{0000000} &
  \bitbox{1}{idx}
\end{bytefield}

\bigskip

The idx field value is 0 for the x index register, and 1 for the y index
register.

\section{branch}

The branch instructions include 11 bits of a 12-bit absolute address of
the branch target. Since instructions must be 16-bit aligned, the branch
target absolute address is always even, so the LSB is omitted from the
instruction encoding.

\bigskip

\begin{bytefield}[endianness=big,bitwidth=1.5em]{16}
  \bitheader{0-15} \\
  \bitbox{5}{opcode} &
  \bitbox{11}{absolute address / 2}
\end{bytefield}

\chapter{Instructions}

\section{opr: operate}

The opr (operate) instruction is a catch-all for miscellaneous functions,
with a bit-mapped opcode containing bits or bitfields encoding those
functions. This allows the combination of multiple functions into a
single instruction. This technique was used in the 1960s and 1970s by
the Digital Equipment Corporation 18-bit and 12-bit minicomputers such
as the PDP-1 and PDP-8, where it was called "microcoding" (similar to
horizontal microcode).

\bigskip

\begin{bytefield}[endianness=big,bitwidth=1.5em]{16}
  \bitheader{0-15} \\
  \bitboxes*{1}{0000} &
  \bitbox{3}{rsv} &
  \bitbox{1}{utx} &
  \bitbox{1}{apc} &
  \bitbox{1}{ret} &
  \bitbox{2}{rot} &
  \bitbox{2}{cy} &
  \bitbox{2}{idx}
\end{bytefield}

\bigskip

\begin{tabular}{ | l | l | l | }
  \hline
  bits  & mnemonic & function \\
  \hline \hline
  11..9 & reserved \\
  \hline
  8     & utx      & UART transmit LSB of accumulator \\
  \hline
  7     & apc      & add accumulator to PC \\
  \hline
  6     & ret      & return from subroutine \\
  \hline
  5..4  & 00 = nop, 10 = rlc, 11 = rrc & rotate left or right with carry \\
  \hline
  3..2  & 00 = nop, 10 = clc, 11 = sec & clear or set carry \\
  \hline
  1..0  & 00 = nop, 10 = tax, 11 = tay & transfer accumulator to index register \\
  \hline
\end{tabular}

\bigskip

Some useful combinations:

\bigskip

\begin{tabular}{ | l | l | l | l | }
  \hline
  opcode & functions & mnemonic & function \\
  \hline \hline
  0x028 & clc rol & lsl & logical shift left accumulator \\
  \hline
  0x038 & clc ror & lsr & logical shift right accumulator \\
  \hline
  0x0c0 & ret apc & retadd & return skipping word count in accumulator \\
  \hline
  0x13c & utx sec rrc & uarttxr & UART transmit LSB of accumulator \\
  \hline
\end{tabular}

\bigskip

The carry operations are executed before the rotate operations.

The tax and tay operations transfer the accumulator contents to the index
register. Since x is an 8-bit register, tax simply performs a copy. The
y register, on the other hand, is a 16-bit register, so in order to allow
loading the entire y register, each use of the tay operation shifts y right
by eight bits and puts the accumulator value into the high 8 bits of y.

For example, if the y register contains 0x1234 and the accumulator contains
0x78, after one tay instruction the y register will contain 0x7812.

The utx operation is specifically intended for use in the uarttxr combination,
which is used for a bit-banged UART transmit output. This was particularly
useful for the RISC-V conformance tests, but may not be suitable for a real
application. The uarttxr instruction should be used once with the LSB of the
accumulator containing zero, to generate a start bit, then the byte to be
transmitted should be loaded into the accumulator, and uarttxr executed nine
more times, with appropriate delays between uarttxr instructions for the
serial bit timing desired.

The apc operation is useful for jump tables. Combined with ret to form the
retadd instruction, it is useful as a return from subroutines that have
multiple exits, with the accumulator specifying a number of instructions
to skip beyond the normal return location.

\section{store: store accumulator to memory}

The store instruction is used to store the contents of the accumulator
register into a byte of memory.

\bigskip

\begin{tabular}{ r c }
  absolute &
  {
    \begin{bytefield}[endianness=big,bitwidth=1.5em]{16}
      \bitheader{0-15} \\
      \bitboxes*{1}{00010000} &
      \bitbox{8}{absolute address}
    \end{bytefield}
  }
  \\
  indexed &
  {
    \begin{bytefield}[endianness=big,bitwidth=1.5em]{16}
      \bitboxes*{1}{00100000} &
      \bitboxes*{1}{0000000} &
      \bitbox{1}{x}
    \end{bytefield}
  }
  \\
  indexed postincrement &
  {
    \begin{bytefield}[endianness=big,bitwidth=1.5em]{16}
      \bitboxes*{1}{00110000} &
      \bitboxes*{1}{0000000} &
      \bitbox{1}{x}
    \end{bytefield}
  }
\end{tabular}

\section{load: load accumulator from memory}

The load instruction is used to load the contents of a byte of memory
into the accumulator register.

\bigskip

\begin{tabular}{ r c }
  immediate &
  {
    \begin{bytefield}[endianness=big,bitwidth=1.5em]{16}
      \bitheader{0-15} \\
      \bitboxes*{1}{01000000} &
      \bitbox{8}{immediate operand}
    \end{bytefield}
  }
  \\
  absolute &
  {
    \begin{bytefield}[endianness=big,bitwidth=1.5em]{16}
      \bitboxes*{1}{01010000} &
      \bitbox{8}{absolute address}
    \end{bytefield}
  }
  \\
  indexed &
  {
    \begin{bytefield}[endianness=big,bitwidth=1.5em]{16}
      \bitboxes*{1}{01100000} &
      \bitboxes*{1}{0000000} &
      \bitbox{1}{x}
    \end{bytefield}
  }
  \\
  indexed postincrement &
  {
    \begin{bytefield}[endianness=big,bitwidth=1.5em]{16}
      \bitboxes*{1}{01110000} &
      \bitboxes*{1}{0000000} &
      \bitbox{1}{x}
    \end{bytefield}
  }
\end{tabular}

\section{and: and memory with accumulator}

The and instruction is used to logically and the contents of a byte of
memory into the accumulator register.

\bigskip

\begin{tabular}{ r c }
  immediate &
  {
    \begin{bytefield}[endianness=big,bitwidth=1.5em]{16}
      \bitheader{0-15} \\
      \bitboxes*{1}{01000001} &
      \bitbox{8}{immediate operand}
      
    \end{bytefield}
  }
  \\
  absolute &
  {
    \begin{bytefield}[endianness=big,bitwidth=1.5em]{16}
      \bitboxes*{1}{01010001} &
      \bitbox{8}{absolute address}
      
    \end{bytefield}
  }
  \\
  indexed &
  {
    \begin{bytefield}[endianness=big,bitwidth=1.5em]{16}
      \bitboxes*{1}{01100001} &
      \bitboxes*{1}{0000000} &
      \bitbox{1}{x}
    \end{bytefield}
  }
  \\
  indexed postincrement &
  {
    \begin{bytefield}[endianness=big,bitwidth=1.5em]{16}
      \bitboxes*{1}{01110001} &
      \bitboxes*{1}{0000000} &
      \bitbox{1}{x}
    \end{bytefield}
  }
\end{tabular}

\section{xor: exclusive or memory with accumulator}

The xor instruction is used to logically exclusive-or the contents of a byte
of memory into the accumulator register.

\bigskip

\begin{tabular}{ r c }
  immediate &
  {
    \begin{bytefield}[endianness=big,bitwidth=1.5em]{16}
      \bitheader{0-15} \\
      \bitboxes*{1}{01000010} &
      \bitbox{8}{immediate operand}
      
    \end{bytefield}
  }
  \\
  absolute &
  {
    \begin{bytefield}[endianness=big,bitwidth=1.5em]{16}
      \bitboxes*{1}{01010010} &
      \bitbox{8}{absolute address}
      
    \end{bytefield}
  }
  \\
  indexed &
  {
    \begin{bytefield}[endianness=big,bitwidth=1.5em]{16}
      \bitboxes*{1}{01100010} &
      \bitboxes*{1}{0000000} &
      \bitbox{1}{x}
    \end{bytefield}
  }
  \\
  indexed postincrement &
  {
    \begin{bytefield}[endianness=big,bitwidth=1.5em]{16}
      \bitboxes*{1}{01110010} &
      \bitboxes*{1}{0000000} &
      \bitbox{1}{x}
    \end{bytefield}
  }
\end{tabular}

\section{adc: add memory to accumulator with carry}

The adc instruction is used to add the contents of a byte of memory
and the carry flag into the accumulator register, and will set the
carry flag based on the result of the addition.

\bigskip

\begin{tabular}{ r c }
  immediate &
  {
    \begin{bytefield}[endianness=big,bitwidth=1.5em]{16}
      \bitheader{0-15} \\
      \bitboxes*{1}{01000011} &
      \bitbox{8}{immediate operand}
      
    \end{bytefield}
  }
  \\
  absolute &
  {
    \begin{bytefield}[endianness=big,bitwidth=1.5em]{16}
      \bitboxes*{1}{01010011} &
      \bitbox{8}{absolute address}
      
    \end{bytefield}
  }
  \\
  indexed &
  {
    \begin{bytefield}[endianness=big,bitwidth=1.5em]{16}
      \bitboxes*{1}{01100011} &
      \bitboxes*{1}{0000000} &
      \bitbox{1}{x}
    \end{bytefield}
  }
  \\
  indexed postincrement &
  {
    \begin{bytefield}[endianness=big,bitwidth=1.5em]{16}
      \bitboxes*{1}{01110011} &
      \bitboxes*{1}{0000000} &
      \bitbox{1}{x}
    \end{bytefield}
  }
\end{tabular}

\section{jump: unconditional jump}

The jump instruction performs an unconditional jump to the target address.

\bigskip

\begin{tabular}{ r c }
  absolute &
  {
    \begin{bytefield}[endianness=big,bitwidth=1.5em]{16}
      \bitheader{0-15} \\
      \bitboxes*{1}{10000} &
      \bitbox{11}{absolute address / 2}
      
    \end{bytefield}
  }
\end{tabular}

\section{call: unconditional subroutine call}

The call instruction pushes the address of the next sequential instruction
into the return register, and branches to the target address.

\bigskip

\begin{tabular}{ r c }
  absolute &
  {
    \begin{bytefield}[endianness=big,bitwidth=1.5em]{16}
      \bitheader{0-15} \\
      \bitboxes*{1}{10001} &
      \bitbox{11}{absolute address / 2}
    \end{bytefield}
  }
\end{tabular}

\bigskip

Glacial normally has only one level of return stack, so nested
subroutine calls will not work.

Returning from a subroutine is accomplished by the ``ret'' function of the
opr (operate) instruction.

\section{skb: skip on bit}

\bigskip

\begin{tabular}{ r c }
  absolute &
  {
    \begin{bytefield}[endianness=big,bitwidth=1.5em]{16}
      \bitheader{0-15} \\
      \bitboxes*{1}{1001} &
      \bitbox{1}{i}
      \bitbox{3}{bit}
      \bitbox{8}{absolute address}
    \end{bytefield}
  }
  \\
  indexed &
  {
    \begin{bytefield}[endianness=big,bitwidth=1.5em]{16}
      \bitboxes*{1}{1010} &
      \bitbox{1}{i}
      \bitbox{3}{bit}
      \bitboxes*{1}{0000000} &
      \bitbox{1}{x}
    \end{bytefield}
  }
  \\
  indexed postincrement &
  {
    \begin{bytefield}[endianness=big,bitwidth=1.5em]{16}
      \bitboxes*{1}{1011} &
      \bitbox{1}{i}
      \bitbox{3}{bit}
      \bitboxes*{1}{0000000} &
      \bitbox{1}{x}
    \end{bytefield}
  }
\end{tabular}

\bigskip

The ``i'' bit determines the polarity of the bit test and the ``bit'' field
determines which bit of the operand is tested, with 0 being the least
significant bit. The skip will occur if the ``i'' bit matches the bit
being tested.

\section{br: conditional branch}

The conditional branch instruction will branch to the target address
if the condition is true. Otherwise execution will continue sequentially.

\bigskip

\begin{tabular}{ r c }
  absolute &
  {
    \begin{bytefield}[endianness=big,bitwidth=1.5em]{16}
      \bitheader{0-15} \\
      \bitboxes*{1}{11} &
      \bitbox{3}{cond}
      \bitbox{11}{absolute address / 2}
    \end{bytefield}
  }
\end{tabular}

\bigskip

The ``cond'' field determines which condition is tested.

\bigskip

\begin{tabular}{ | c | l | }
  \hline
  cond & condition \\
  \hline \hline
  0    & accumulator non-zero \\
  \hline \hline
  1    & accumulator zero \\
  \hline \hline
  2    & no carry \\
  \hline \hline
  3    & carry \\
  \hline \hline
  4    & no external interrupt \\
  \hline \hline
  5    & external interrupt \\
  \hline \hline
  6    & no clock tick \\
  \hline \hline
  7    & clock tick \\
  \hline \hline
\end{tabular}

\end{document}
